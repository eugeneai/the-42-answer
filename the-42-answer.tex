
\documentclass[12pt,a4paper]{extarticle}
\usepackage{ifxetex}
\usepackage{ifluatex}
\usepackage{graphicx}
%\usepackage[english,russian]{babel}
\usepackage{polyglossia}

\ifluatex
  \usepackage{pdftexcmds}
  \makeatletter
  \let\pdfstrcmp\pdf@strcmp
  \let\pdffilemoddate\pdf@filemoddate
  \makeatother
\fi
\usepackage{svg}
% \setsvg{inkscape={"/usr/bin/inkscape"= -z -C}}
\setsvg{inkscape = inkscape -z -D}

% \fontspec{Gentium}
\setmainfont{PT Sans}
\setmonofont{Fira Mono}
\setdefaultlanguage{russian}
\setotherlanguage{english}

\begin{document}

\section{Проблематика MDE}
\label{sec:problem-mde}

Основная проблема разработки программного обеспечения (ПО) --- это сложность,
связанная с большим количеством взаимодействующих гетерогенных компонент в
рамках одного программного комплекса.  В основе каждого компонента находится
модель, являющаяся продуктом анализа предметной области (см. рис.
\ref{fig:mde-gen-schema}), которая должна изменяться в соответствии с изменением
предметной областью.

Процесс внесения изменений в ПО представляет собой исправление программного
кода, при этом отображаемая им модель остается без исправлений.  Основная задача
MDE --- это использование интеллекта разработчика на этапе моделирования, а не
кодирования.

\begin{figure}[htbp]
  \centering
%  \includesvg{mde-general}
  \caption{Общая схема MDE}
  \label{fig:mde-gen-schema}
\end{figure}

[Теоретико-категорные представления категории метамодели]

\section{Трансформация моделей (MDA)}
\label{sec:mda-transform}

По большому счету MDE не интересует процесс трансформации, т.к. интересует
больше процесс внесения изменений (актуализация категории метамодели).  Но в
целом процесс представляет собой последовательное уточнение описания модели до
конкретных модулей, реализованных в конкретной среде программирования.

\section{Подходы к внесению изменений в модели}
\label{sec:mde-conversions}

В настоящее время выделяются два подхода к реализации трансформации ---
[физическое] объединение моделей и распространение изменений.  В первом подходе
на каждом этапе объединения из двух моделей строится одна, при этом общие
концепты сливаются в один.  Второй подход базируются на расслоении категории
метамоделей на отдельные модели, связанные друг с другом через функторы.  Эти
функторы в нотации полисистемного анализа и синтеза реализуют интерпретации
объектов и стрелок (морфизмов).

Процесс объединения моделей (первый случай) состоит из двух шагов --- а) создание
спецификации объединения, и этот этап является творческим, б) собственно
объединение, представляющее собой выполнение алгебраических операций над
моделями с целью построить общий копредел диаграммы (категорию).  Случай
аналогичный этому рассматривается в диссертации д.ф.--м.н. Ковалева Сергея
(ИДСТУ СО РАН выступал в качестве ведущей организации в 2012 году).  Основное
достоинство подхода --- объединение моделей - первый шаг в трансформации: в
результате объединения создается модель модуля, где все функции интегрированы в
модуль, что влечет более оптимизированный программный код (далекий от MDE пример
--- LLVM).  У С. Ковалева в качестве базовой методики проектирования выступало
аспект-ориентированное программирование.

Во втором случае трансформация моделей представляет собой процесс восстановление
сквозной интерпретации концептов и стрелок в расслоении, в случае, если одна из
моделей подверглась изменению.  В частности таким слоем выступает реализованный
в системе программирования модуль.  Актуализация [трансформация] представляет
собой преобразование пары морфизмов (межслойная интерпретация;измененная
структура в слое, рис. ...) в новую пару морфизмов и распространение этой
процедуры на другие слои.  Достоинство подхода --- более естественное для
человека представление метамодельной категории (в виде слоев) и локальная
процедура изменений.

В обоих случаях присутствуют этапы, где необходимо задействовать разработчика.
В первом случае --- формирование спецификации объединения, во втором --- изъятие
недостающей информации при актуализации эпиморфизмов и мономорфизмов.

\section{Построение метамодельной категории}
\label{sec:mmod-construction}

На первоначальном этапе MDE существует проблема первоначального построения
метамодельной категории --- необходимо построить функторы между моделями, т.е.
морфизмы в категории метамодели.  Для этого в предметной области необходимо
выделить концепты и отношения между ними, представить их в виде концептуальной
модели, например, в виде полисистемы онтологий, связать эти концепты и отношения
со структурой остальных моделей.

Разработанный процедуры распознавания концептов на примере рабочих и учебных
программ вуза --- это примеры реализации этой задачи.  По набору примеров
(абзацев), их характеристических свойств, определяется концепт, связь между
концептами, в нашем случае --- это последовательности и зависимости (структурные
и семантические), затем, задаются через отображения в другие слои.  В качестве
трансформации выступает процедура представление системы модулей, всяческие
проверки на полноту (относительно требуемых в слоях отношений между
известными/распознанными концептами), генерация новых версий представлений (по
новым ФОС) и т.п.

\section{Методика построения онтологической модели предметной области}
\label{sec:technique-onto}

Здесь необходимо доделать задачу анализа процесса разработки программы,
фиксируемого в репозитариях типа GIT.  Задача состоит в том, чтобы в сообщениях
программистов выделить концепты и построит в слоях семантическую интерпретацию
через указание кусков кода, реализующих этот концепт.

\end{document}


%%% Local Variables:
%%% TeX-master: t
%%% End:
